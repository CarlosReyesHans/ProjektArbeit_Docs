\chapter{State of the art}\label{cha:state}

%% Proposed roadmap
%
%   The use of RT in the industrial environments
%   Current standards IEEE and IEC and the TSN
%           Roadmap of standards
%           RTE protocols and OPC UA   
%           How are they related
%  
%   -----------State of the art--------------------------------
%   Applications with RTE and OPC UA in Hard Real Time applications     
%   A brief overview about the RT tools in robotics
%           The importance of EtherCAT as an open RTE protocol within robotics
%   Comparison of openess
%   EtherCAT introduction (selected by Hans Robot)
%           Focus on EtherCAT XoE <<THIS ONLY NEEDS TO BE MENTIONED
%------------------Out of scope--------------------------------------
%   HW ASICS and stuff like that
%   SW Stacks for EtherCAT development and other RTEs
%

This chapter introduces some current applications mainly focused on robotics, since this area is closely
related to the environment with which the ACB will be interacting. Robotics sees various advantages
from the RT communication protocols, when it comes to integrate motion controllers and 
any other industrial peripheral. Afterwards, an overview about industrial development frameworks is 
given, yet focused not on the RT interfaces, but its specific software. The latter is of great importance,
for the Real Time Operative Systems (RTOS) are a corner stone for embedded systems that need to provide a 
deterministic service within their environment.


\section{Current applications}\label{sec:applications}

As rapidly mentioned in Sect.~\ref{sec:standards}, the SERCOS motion control interface has been standardized within the 
CPFs of IEC 61784-Part 2. Furthermore, it has been even integrated to EtherCAT as a compatible CP. 
This service is available within the DLL and AL and is called Servo Drive Profile over EtherCAT  (SoE), 
which provides access to motion controllers under 
the SERCOS specifications and, consequently, offers interoperability within its own RT features and the latter's hard RT 
capabilities.  
An example of this compatibility is presented in ~\cite{ecat_sercos}, %Motion Control System using SERCOS over EtherCAT
it shows that jitter of 30 microseconds is feasible in a control loop while the Master uses the SoE service.

Another interesting application has been the characterization of an EtherCAT Master within a RT control 
loop for Servo Motors, which run CAN devices over EtherCAT (CoE service). The implementation of the Master 
device ran on different open source Real Time Operating Systems (RTOS) based on Linux, 
namely, Xenomai and Linux with the \emph{RT\_PREEMPT} patch. It was concluded that both of the approaches 
were capable to achieve update periods of \SI{1}{\milli\second} 
, and an average jitter of \SI{1.5}{\micro\second}. Moreover, Xenomai could averagely achieve execution times around \SI{100}{\micro\second}; 
the mentioned data can be reviewed in ~\cite{ecat_xenomai}.%Real-time Servo Control using EtherCAT Master on Real-time Embedded Linux Extensions. 
It is worth to mention that the EtherCAT Master features were available in both RTOS kernels, 
for the IgH EtherCAT Master stack was running on top of them. This open source stack will be commented in ~\ref{sec:openness}.

The characterization and optimization of performance for different RTE profiles within TSN is a currently expanding topic, 
as the TSN standards and the RTE commissions are still working together. In ~\cite{tsn_and_sdn} %Integrated Industrial Ethernet Networks: Time-sensitive Networking over SDN Infrastructure for mixed Applications
are presented simulations of TSN topologies with EtherCAT and SERCOS data frames, where the Quality of Service (QoS) 
is addressed and evaluated through the usage of Software-defined Network (SDN) switches. 
The approach of this project is to test different scheduling features given fixed cycle times for the data frames, 
which were proposed to be similar to the current real industrial applications
in both technologies. In this manner, the importance of an unified network that supports different protocols is highlighted, 
but further research in this topic, including tests with other RTE data frames are still to be researched.

Besides robotics, a recent industrial application concerning CBD extraction equipment for high-performance large-scale processing,
implemented distributed control and monitoring based on EtherCAT open protocol. This article can be seen in ~\cite{ecat_industrial}.

Addressing the usage of open source tools, such as OS and RTE Protocols, for development of complex robotic systems, in ~\cite{ecat_motionplanning} %Motion Planning for Quadrupedal Locomotion: Coupled Planning, Terrain Mapping and Whole-Body Control
is presented a \emph{Motion Planning for Quadrupedal Locomotion}. This is roughly composed, besides the hydraulic actuators, mechanics and other peripherals, of two PCs on board with RT capabilities and shared memory. 
RT Linux (Xenomai) runs on both of them and take care of different levels of the control threads at two different rates depending on the tasks, 
namely \SI{1}{\kilo\hertz} and \SI{250}{\kilo\hertz}. The former rate is used
for communicating with the motor controllers over EtherCAT interfaces.

Currently, Han's Robot Germany GmbH focuses on enhancing robots’ cognitive abilities by developing in the fields of environment perception, 
drive technologies, 
control theory, material science, mechanical design and artificial intelligence. Interfaces within the robotic system rely on various
industrial protocols to make its interoperability one of the key features. For instance, current motor drivers
are linked over internal EtherCAT chain to the main controller.

The above mentioned applications are just a tiny number of examples that shows the importance of an already standardized open industrial communication protocol, 
within a broad set of fields that cannot be completely covered in the scope of this document. Nevertheless, it paves the road to understand why generating the know-how 
to any of the mentioned technologies, represents a high-impact resource for any research or development group, regardless of its commercial or academic purposes.

\section{An overview about the RT capable SW in robotics}

As mentioned in the previous section, several resources and examples showed the current usage of RT
open source software and its community. Since this Research Project has a goal of introducing the reader a 
roadmap for RTE communication interfaces and 
its applications, this section was added to summarize the RT software for development in robotics.

The usage of middlewares within the field of robotics is growing and it relies on \emph{robot software} that 
exists between the application and an RTOS, as detailed in the following articles ~\cite{ecat_xenomai} and ~\cite{ecat_motionplanning}.%the robot examples.

A list of requirements is suggested in ~\cite{middleware_industrial} %2020-Real-Time Robot Software Platform for Industrial Application 
to address the mentioned middlewares and how to consider them \emph{Real Time Robot Software Platforms}.  
The list is as follows and it is useful to start getting familiar with the capabilities 
and features of the so-called Robot Software: 
\begin{enumerate}
    \item Data exchange support.
    \item Real time support (strict periodic execution and sporadic events support).
    \item Thread and process types for user defined programs support.
    \item Easy configuration of applications (robot control SW, PLC SW, vision inspection SW, non-real-time SW, etc.)
    \item Multiple periods for scheduling.
    \item Threads or processes running in the same period are classified by priority.
    \item Check and handle the event through the event handler.
\end{enumerate}

Common names for different projects aiming to create these development frameworks are the following: Common Object Request Broker Architecture
(CORBA), Real-Time CORBA (RT-CORBA), Data Distribution Service (DDS), OPC UA, Open Platform for Robotic Services (OPRoS), 
Open Robotics Technology Middleware (OpenRTM), Open Robot Control Software (OROCOS), 
and Real-Time Middleware for Industrial Automation devices (RTMIA). Further comments and a comparisson between their features
can be seen in the previously referenced paper. As to what concerns to this document, only some of them will be roughly commented
as they ended up being somehow related to the RTE profiles. For more information review ~\cite{middleware_xbotcore}.

\begin{description}
    \item[OPC UA] As frequently mentioned before, this is an open standard for data sharing among nodes within industrial networks and has
    been considered in some projects related to robotics. Nevertheless, it is important to highlight that this is not considered a full 
    middleware, since it only provides a protocol to control the exchange of data between nodes, a good degree of reliability and security.
    However, it does not provide RT capabilities to the system only compatibility. Hence, it needs an operative system and the consequent 
    lower layers capable of RT scheduling and communication, concerning the latter the TSN set for protocols is an example.
    \item[ROS/OROCOS/OpenRTM] These are projects that aim to create a suitable middleware for robots by implementing Xenomai or Linux operating systems. 
    ROS prioritizes the final user, avoiding in the way some fine-grained features due to its difficulty, therefore having sometimes
    issues to meet the hard RT requirements. Whereas OROCOS has further improved its compatibility, similarly to OpenRTM.
    \item[CODESYS and TwinCAT] To fully meet compatibility with the industry, the so-called PLC Software has been also used in open robotics.
    These applications essentially need to run both, the robot functional blocks and the robot tasks. For further details on it the following
    references can be reviewed ~\cite{middleware_xbotcore} and ~\cite{middleware_xbotcloud}. %referemces from the italian robot Xrobotcore
    \item[xbotcore] This is an attempt to provide of a highly compatible open middleware for industrial robotics, it runs over the Xenomai and 
    uses a SOEM stack to interface with any compatible industrial device, recall ~\ref{sec:openness}. % XBotCore: A Real-Time Cross-Robot Software Platform 
    Applications have been already mentioned in ~\ref{sec:applications}, which have reached control loops down to 1khz for 33 axes. More details about the 
    latter application in ~\cite{middleware_xbotcore}.
    %\item[RTMIA] RTMIA middleware + Linux or Xenomai but used open PLC running parallely[?] << This needs further reading [?]
\end{description}


The previous information was presented only to draw an idea for the non-familiar reader about the applications and, since this topic is in ongoing development 
and, furthermore, many other platforms are addressing similar challenges; the reader is invited
to go deeper into these topics, for instance, by reviewing this resource ~\cite{middleware_industrial}. %add the resouce fromt he open community 

\section{Approaching openness within the RT protocols}\label{sec:openness}

Among the industrial standards mentioned in ~\ref{sec:standards}, there are some related initiatives to include a certain degree of open source
software to improve the development of applications. The following is a brief list of a few interesting references to them. 
However, as expected, most of the software stacks
for industrial communication systems are commercial and provided by third-party companies. 

%90/100
\begin{description}
    \item[OSADL] Open Source Automation Development Lab eG (OSADL):It is a German group that intends to lead the development of open source development
    for industrial automation. Closely related in the developing of OPC UA and other Linux features for industrial applications.
    \item[open62541] Within the official scope of OPC UA, there is this Certified SDK project that is within its second phase, at which it is expected 
    from the research and industrial community to develop applications to test its performance. Moreover, as the TSN specification is of huge importance, 
    a set of enhancements for the open62541 project were developed by Fraunhofer IOSB and series of patches for the Linux kernel have been released to 
    make it an RT compatible. To review the overview of the project, visit ~\cite{open62541_homepage}. %https://www.osadl.org/OPC-UA-over-TSN.opcua-tsn.0.html 
    The OPC UA is developed under GPL 2.0 license and due to its current phase implies a further adaptation for the physical node, e.g., ARM arquitechtures 
    to make them compatible with the mentioned patches. 
    \item[SOEM/SOES] RT Labs Industrial development group focused on Software Stacks for industrial protocols. Among their commercial communication stacks
    there are software stacks under GPL for EtherCAT Master and Slave devices SOEM/SOES. More details about these stacks can be found in ~\cite{soesm_homepage}. %https://rt-labs.com/product/soes-ethercat-slave-stack/ 
    \item[Sercos Stacks] Sercos III technology is able to be operated in a common TSN-based network, % Ethernet TSN heralds a new era of industrial communication
    since its development group has been working closing together with the TSN group; a more detailed scope can be seen in ~\cite{sercos_tsn}. 
    They also made available open source software dedicated for development of master and slave devices, namely: Common Sercos Master API (CoSeMa),
    Sercos Internet Protocol Services (IPSS), and The Sercos III SoftMaster, the latter even allows the host to use any standard Ethernet controller. %https://www.sercos.org/technology/implementation/driver-software/. 
    It is important to mention that there are testing tools to certificate those devices and achieve a Safety Integrity Level 3 (SIL-3). A description of the
    different available libraries is provided in ~\cite{sercos_stacks}.
    \item[EtherNet/IP Stacks] There are several commercial stacks that comply with the Open DeviceNet Vendors Association's (ODVA) EtherNet/IP specification. 
    OpEner is an open source alternative which targets PCs with a POSIX operating system and a network interface. 
    Integration examples are provided only in Linux and Windows in ~\cite{opener_stack}; %http://eipstackgroup.github.io/OpENer/index.html
    nevertheless, a variation for embedded systems has been presented for an STM32 microcontroller. In the mentioned project, more tools had to be adapted, for instance,
    STM32F4x7 Ethernet Driver v1.1.0, lwIP v1.4.1 (TCP/IP Stack), MicroHTTP v5.1.0.1, a patched version of OpENer v1.2., among others. A detailed description
    of the requirements can be reviewed in ~\cite{opener_stm32}%https://www.emb4fun.de/archive/chibiosopener/index.html
    \item[IgH EtherCAT Master] This is a bundle of libraries to give a Linux host (LinuxCNC for example) EtherCAT Master features, it is developed under the GPLv2 license. 
    An interesting example of this open source resource within an Airbus Test Rig can be reviewed in the following reference ~\cite{ecatstack_igh}. %https://etherlab.org/en/success/airbus-testrig.php
\end{description}

As the scope of this Research Project is only focused on the industrial communication profiles capable of RT and, so far has been clear how 
the EtherCAT is a reliable one, yet open and significantly considered in the industry ---recall Sect. ~\ref{sec:applications}. 
Hence, it makes sense to invite the reader to read the introduction to the protocol itself in the Appendix ~\ref{sec:ecat_protocol}. 
This way it is easier to go sensibly to the implementation of what is one of the basic chain-elements in what could become a very complex application: 
an EtherCAT Slave device with open source elements.

