\chapter{SMs tests}








% \begin{figure}[ht]
%     \centering
%     \subfigure[Synchronization state machine]{\label{subfig:ecat_sm}{
%         %General state machine 80/100
%         \begin{tikzpicture}[->,>=stealth']
%         % Position of QUERY 
%         % Use previously defined 'state' as layout (see above)
%         % use tabular for content to get columns/rows
%         % parbox to limit width of the listing
%         \node[state,
%             text width=3.2cm 	
%             ] (E_CONFIG) 
%         {\begin{tabular}{l}
%             \textbf{Config}\\[.1em]
%             \parbox{4cm}{
%             \textbf{entry:}\\
%             $spi\_init()$\\
%             $open\_soesPort()$\\
%             \textbf{exit:}
%             }
%         \end{tabular}};

%         % STATE START 
%         % Use previously defined 'state' as layout (see above)
%         % use tabular for content to get columns/rows
%         % parbox to limit width of the listing
%         \node[state,
%             text width=3.2cm,
%             above of = E_CONFIG,
%             node distance = 3.5cm,
%             anchor = center] (E_START) 
%         {\begin{tabular}{l}
%             \textbf{Start}\\[.1em]
            
%         \end{tabular}};


            
%         % State: ACK with different content
%         \node[state,    	% layout (defined above)
%             text width=3.2cm, 	% max text width
%             yshift=2cm, 		% move 2cm in y
%             right of=E_CONFIG, 	% Position is to the right of QUERY
%             node distance=5cm, 	% distance to QUERY
%             anchor=center] (E_CHCK) 	% posistion relative to the center of the 'box'
%         {%
%         \begin{tabular}{l} 	% content
%             \textbf{Check comm}\\[.1em]
%             \parbox{2.8cm}{
%                 \textbf{entry:}\\
%                 $timeout\_start()$
%                 \textbf{exit:}\\
%                 $resume\_SOES()$
%                 }
%         \end{tabular}
%         };
        
%         % STATE E_WAIT
%         \node[state,
%         text width=3.2cm,
%             %below of=ACK,
%             right of=E_CHCK,
%             yshift=-2cm,
%             node distance=5cm, 
%             anchor=center ] (E_WAIT) 
%         {%
%         \begin{tabular}{l}
%             \textbf{Waiting SOES}\\[.1em]
%             \parbox{2.8cm}{
%                 \textbf{entry:}\\
%                 $osWaitEvent()$\\
%                 \textbf{exit:}
%                 }
%         \end{tabular}
%         };
        

        
%         %STATE CONNECTED
%         \node[state,
%         text width = 3.2cm,
%         below of=E_WAIT,
%         xshift= 2.5cm,
%         node distance=4.5cm,
%         anchor=center] (E_CONNECTED) 
%         {%
%         \begin{tabular}{l}
%         \textbf{Connected}\\[.1em]
%         \parbox{4cm}{
%             \textbf{entry:}\\
%             $osResume(SOES)$\\[.1em]
%             $update\_state()$\\
%             \textbf{exit:}
%             }
%         \end{tabular}
%         };

%         % STATE E_FAULT
%         \node[state,
%             text width=3.2cm,
%             below of=E_WAIT,
%             xshift=-2.5cm,
%             node distance=4.5cm,
%             anchor=center] (E_FAULT) 
%         {%
%         \begin{tabular}{l}
%         \textbf{Fault}\\[.1em]
%         \parbox{4cm}{
%             \textbf{entry:}\\
%             $notify\_error()$\\
%             $osThreadRemove()$\\
%             \textbf{exit:}
%             }
%         \end{tabular}
%         };

%         %STATE RESTART
%         \node[state,
%         text width = 3.2cm,
%         left of=E_FAULT,
%         node distance=5cm,
%         %xshift = 2cm,
%         anchor=center] (E_RESTART) 
%         {%
%         \begin{tabular}{l}
%         \textbf{Restart}\\[.1em]
%         \parbox{4cm}{
%             \textbf{entry:}\\
%             $deInit_SPI()$\\[.1em]
%             $restartSOES$\\
%             $delay()$
%             }
%         \end{tabular}
%         };

%         % draw the paths and and print some Text below/above the graph
%         \path
%             (E_START)       edge node[anchor=center,left]{$os\_ready$}  (E_CONFIG) 
%             (E_CONFIG) 	    edge[bend left=20]  node[anchor=north,above]{$port\_ok$} (E_CHCK)
%             (E_CHCK)     	edge[bend left=20] node[anchor=north,right]{$Resume!$} (E_WAIT)
%             (E_WAIT)       	edge[bend left=10]  node[anchor=center,right]{$Comm?$}     (E_CONNECTED)
%             (E_WAIT)       	edge[bend right=10] node[anchor=center,left]{$soes\_timeout$}                                         (E_FAULT)
%             (E_CONNECTED)   edge            node[anchor=north,above]{$soes$}     
%                                             node[anchor=south,below]{$timeout$}(E_FAULT)
%             (E_CONNECTED)   edge[loop below]  node[anchor=center,below]{$update\_loop$}                   (E_CONNECTED)
%             (E_FAULT)  	    edge            node[anchor=north,above]{$thread$} 
%                                             node[anchor=center,below]{$removed?$}   (E_RESTART)
%             (E_RESTART)  	edge[bend left=30]  node[anchor=center,left]{$delay\_done$}                                          (E_CONFIG)
%             ;

%         \end{tikzpicture}
%     }}\hfill
%     \subfigure[SOES application state machine]{\label{subfig:soes_sm}{
%         %SOES state machine 80/100
%         \begin{tikzpicture}[->,>=stealth']
%         % STATE 1 SOES
%         % Use previously defined 'state' as layout (see above)
%         % use tabular for content to get columns/rows
%         % parbox to limit width of the listing
%         \node[state,
%             text width=3.2cm 	
%             ] (SOES_INIT1) 
%         {\begin{tabular}{l}
%             \textbf{Init1}\\[.1em]
%             \parbox{4cm}{
            
%             $lib\_init()$\\
            
%             }
%         \end{tabular}};
            
%         % STATE SOES INIT 2
%         \node[state,    	% layout (defined above)
%             text width=3.2cm, 	% max text width
%             %yshift=2cm, 		% move 2cm in y
%             below of=SOES_INIT1, 	
%             node distance=3cm, 	% 
%             anchor=center] (SOES_INIT2) 	% posistion relative to the center of the 'box'
%         {%
%         \begin{tabular}{l} 	% content
%             \textbf{Init2}\\[.1em]
%             \parbox{2.8cm}{
%                 $suspend\_thread()$\\
%                 }
%         \end{tabular}
%         };

%         % STATE 0 SOES
%         % Use previously defined 'state' as layout (see above)
%         % use tabular for content to get columns/rows
%         % parbox to limit width of the listing
%         \node[state,
%             text width=3.2cm,
%             right of = SOES_INIT1,
%             node distance = 5cm,
%             anchor = center] (SOES_INIT0) 
%         {\begin{tabular}{l}
%             \textbf{Start}\\[.1em]
%             \parbox{4cm}{
            
%             }
%         \end{tabular}};
        
%         % STATE SET TIMER
%         \node[state,
%         text width=3.2cm,
%             %below of=ACK,
%             right of=SOES_INIT2,
%             %yshift=-2cm,
%             node distance=5cm, 
%             anchor=center ] (SOES_TIMER) 
%         {%
%         \begin{tabular}{l}
%             \textbf{Timer}\\[.1em]
%             \parbox{2.8cm}{
%                 $osGetTick()$\\
%                 $osStartTimeout()$
%                 }
%         \end{tabular}
%         };
        

        
%         %STATE SLAVE LOOP
%         \node[state,
%         text width = 3.2cm,
%         right of=SOES_TIMER,
%         %xshift= 2.5cm,
%         node distance=5cm,
%         anchor=center] (SOES_SLAVE) 
%         {%
%         \begin{tabular}{l}
%         \textbf{Slave loop}\\[.1em]
%         \parbox{4cm}{
%             $eca\_slv()$\\
%             }
%         \end{tabular}
%         };

%         % STATE SOES_RESET TIMEOUT
%         \node[state,
%             text width=3.2cm,
%             below of=SOES_SLAVE,
%             xshift=-2.5cm,
%             node distance=3cm,
%             anchor=center] (SOES_RESET) 
%         {%
%         \begin{tabular}{l}
%         \textbf{Timeout Reset}\\[.1em]
%         \parbox{4cm}{
%             $reset\_Timeout()$\\
%             $osDelay()$\\
%             }
%         \end{tabular}
%         };

        
%         draw the paths and and print some Text below/above the graph
%         \path
%         (SOES_INIT0)    edge node[anchor=center,above]{$Resume?$} (SOES_INIT1) 
%         (SOES_INIT1) 	edge  node[anchor=south,right]{$Comm!$}  (SOES_INIT2)
%         (SOES_INIT1)    edge [loop above]   node[anchor=east,above]{$no\_comm\_loop$} (SOES_INIT1)
%         (SOES_INIT2)    edge [loop below]   node[anchor=south,below]{$no\_eventFlag$} (SOES_INIT2)
%         (SOES_INIT2)    edge node [anchor=center,above]{$Resume?$} (SOES_TIMER)                                          (SOES_TIMER)
%         (SOES_TIMER)    edge                   (SOES_SLAVE)
%         (SOES_SLAVE)    edge [loop above]       node[anchor=center,above]{$no\_comm\_loop$}       (SOES_SLAVE)
%         (SOES_SLAVE)    edge [bend left = 50]   node[anchor=center,right]{$esc\_updated$}   (SOES_RESET)
%         (SOES_RESET)    edge [bend left = 50]   node[anchor=center,left]{$delay\_done$}   (SOES_TIMER)
%         ;

%         \end{tikzpicture}
%     }}
%     \caption{State machines for EtherCAT slave functionality} %EtherCAT Device Protocol poster from EtherCAT resources
%     \label{fig:syncmodes}
% \end{figure} 
  